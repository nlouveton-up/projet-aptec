\documentclass[12pt,a4paper]{article}
\usepackage[utf8]{inputenc}
\usepackage[french]{babel}
\usepackage[T1]{fontenc}
\usepackage{geometry}
\usepackage{array}
\usepackage{booktabs}
\usepackage{longtable}

\geometry{margin=2cm, landscape}

\title{\textbf{Référentiel de compétence éthique et technologie émergentes}}
\author{Nicolas Louveton}
\date{Janvier 2026}

\begin{document}

\section*{Référentiel de compétence éthique et technologie émergentes}

\noindent Nicolas Louveton, Janvier 2026. Adapté de Bruneault, F., Sabourin Laflamme, A., \& Mondeaux, A. (2020). \textit{Former à l'éthique de l'IA en enseignement supérieur: Référentiel de compétence.}

\begin{longtable}{|>{\raggedright}p{0.12\textwidth}|>{\raggedright}p{0.21\textwidth}|>{\raggedright}p{0.21\textwidth}|>{\raggedright}p{0.21\textwidth}|>{\raggedright\arraybackslash}p{0.21\textwidth}|}
\hline
& \textbf{Aspects techniques} & \textbf{Dilemmes moraux} & \textbf{Contexte sociotechnique} & \textbf{Cadres normatifs} \\
\hline
\endfirsthead

\hline
& \textbf{Aspects techniques} & \textbf{Dilemmes moraux} & \textbf{Contexte sociotechnique} & \textbf{Cadres normatifs} \\
\hline
\endhead

\hline
\endfoot

\textbf{Être} &
Repérer les situations ou le fonctionnement technique du système soulève des question éthiques compte tenu de sa place dans le quotidien &
Déceler et identifier les dilemmes moraux associé à l'usage d'un système technique dans le cadre d'un pluralisme moral &
Repérer les problématiques sociotechniques liées à la conception et au déploiement de la technologie &
Identifier les dimensions pertinentes prévues par les différents cadres normatifs contribuant à l'encadrement des conduites liées à la conception, au déploiement et l'utilisation de la technologie \\
\hline

\textbf{Agir} &
Mobiliser ses connaissances et compétences technique afin de réaliser une évaluation réflexive et critique du système &
Problématiser le ou les dilemmes moraux en mobilisant des connaissances et compétences sur les questions de moralité (cadre philosophique) &
Repérer les problèmes particuliers liés aux enjeux politiques, sociaux, économiques, culturels et environnementaux liés à la technologie &
Mobiliser les différents cadres normatifs qui s'appliquent aux problèmes, évaluer leur pertinence et dégager les tensions qui découlent de leurs interactions. \\
\hline

\textbf{Dialoguer} &
Exposer son point de vue sur les caractéristiques du système dans le contexte de donnée afin de susciter la délibération et de trouver une solution. &
Évaluer sa propre position en la confrontant à celle d'autres parties prenantes afin de délibérer sur les actions requises dans le contexte du dilemme moral &
Discuter les différentes perspectives concernant les questions sociotechniques pour anticiper les impacts sociétaux des différentes actions possibles en contexte &
Expliquez la priorisation des normes qui justifient sa propre position et apprécier celle qui guide la position des autres parties prenantes, délibérer pour trouver des solutions potentielles \\
\hline

\end{longtable}



\end{document}
