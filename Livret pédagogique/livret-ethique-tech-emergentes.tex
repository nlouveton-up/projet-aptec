\documentclass[12pt,a4paper]{book}

% Packages pour le français
\usepackage[utf8]{inputenc}
\usepackage[T1]{fontenc}
\usepackage[french]{babel}

% Packages utiles
\usepackage{graphicx}
\usepackage{hyperref}

\usepackage{array}
\usepackage{booktabs}
\usepackage{longtable}

% Métadonnées
\title{Scénarios pédagogiques pour une reflexivité éthique et les technologies émergentes}
\author{Nicolas Louveton}
\date{\today}

\renewcommand{\familydefault}{\sfdefault}

\begin{document}

% Page de titre
\maketitle

% Table des matières
\tableofcontents

% Chapitres
\chapter{Introduction}

\begin{verse}
	\ldots \emph{du seul philosophe, parmi les artisans, les lois sont durables et les actions droites et belles} \ldots
\end{verse}

Cette citation, attribué à Aristote, me semble particulièrement appropriée ici. On note que le philosophe est ici -- est c'est assez surprenant quand on songe à l'image que nous avons des philosophes comme de purs intellectuels -- un artisan, c'est-à-dire que d'une certaine façon, il construit son action, avec les outils de la raisons afin d'atteindre une forme de vérité qui s'identifie à la vertue.

Ce manuel est construit exactement dans cette idée : Il ne s'agit pas de transmettre une doctrine morale ou une éthique toute faite, prête à porter. Au contraire, il s'agit de fournir les outils qui permettent aux étudiant d'élaborer leur propre reflexion, argumentée bien sûr, qui se tienne debout pour lui-même mais aussi qu'il puisse la communiquer, échanger et la faire évoluer au contact des autres.

Les défis éthiques liés aux technologies se multiplient et avancent tellement rapidement : l'arrivée des systèmes génératifs comme ChatGPT a bousculé notre société de l'information ou l'écrit et le document sont essentiels pour s'informer, se construire un point de vue, évaluer, etc. Les régulateurs ne peuvent que suivre dans un second temps. Or, les intrications se multiplient : écologie, sécurité, désinformation, perte de compétence voire de travail, etc.


J'ai pris la décision en tant qu'enseignant-chercheur d'éviter deux pièges : le premier est de fuir absolument le contact avec ce qui me dérange (la génération automatique de documents par IA) ; la seconde, de penser pouvoir former au prochain métier à la mode qui sera déjà obsolet dans quelques mois (pensons au prompt engineering par exemple). Il faut donc viser des compétences plus abstraites : il nous faut nous doter des outils qui serviront de boussole aux étudiants. 

Malgré les évolutions du contexte,le futur citoyen, le futur professionnel devra pouvoir sortir sa boussole et s'y retrouver par un cheminement sûr, du moins raisonnable. Ce sont ces outils-là que nous visons à créer par l'intermédiaire de ces scénarisations pédagogiques. Car il faut sans cesse réactualiser, énacter (voir Varela), son approche éthique en la situant dans un contexte global.

Pour ce faire, je m'appui sur les travaux de nos collègues Québécois. XXX. Dans leur article, ils proposent un référentiel de compétence, composé de deux dimensions. YYYY.

% Bruneault, F., Sabourin Laflamme, A., \& Mondeaux, A. (2022). \textit{Former à l'éthique de l'IA en enseignement supérieur: Référentiel de compétence.}

Dans la continuité de ce que ZZZ ont proposé, j'étends leur travail sur des thèmes que j'enseigne spécifiquement en cours à l'université. Il s'agit de formaliser un processus pour animer une séance, produire des reflexions et des débats et enfin évaluer le résultat des travaux en termes de compétences.

Ce travail est de nature ouverte car il est dispoinible librement et peut être continué, étendu, amélioiré, etc. Tout retour, commentaire et collaborations sont les bienvenus.


\section*{Présentation des thématiques couvertes dans ce document}



\section*{Présentation du référenciel de compétence}

\input{chapitres/referentiel-competences.tex}

\chapter{Thème : science-fiction et innovation}



% ============================================
% DESCRIPTION DU COURS SCI-FI IHM et Innov
% ============================================

\section{Objectif}

L'objectif de ce cours est d'initier les étudiants au domaine de la conception d'interface humain-machine en utilisant les œuvres de science-fiction comme illustration. Ces œuvres non seulement ont été précurseurs dans leur façon de décrire les évolutions technologiques mais aussi pour mettre en lumière des tensions éthiques en rapport avec ces évolutions. Il s'agira donc également d'amener à une réflexion individuelle et collective sur ces tensions ainsi que le positionnement que l'on peut adopter.

\section{Compétences visées}

\begin{itemize}
  \item Savoir identifier et analyser les choix de conception d'une interface innovante/futuriste en s'appuyant sur une grille d'analyse
  \item Comprendre les implications sociotechniques, morales et réglementaires de cette interface
  \item Savoir se positionner personnellement et argumenter en groupe
\end{itemize}

% ============================================
% TRAME DES ACTIVITÉS
% ============================================

\section{Trame et activités}

\subsection{Présentation (10 minutes)}

Introduction générale au cours et présentation des objectifs.

\subsection{Activité 1 : expérience (20 minutes)}

Nous regardons ensemble une courte œuvre ou un extrait d'œuvre de science-fiction (par ex. la vidéo \emph{Hyper Reality}).

Individuellement, vous notez toutes vos réactions, ressentis, idées.

Ensuite, échangez avec votre voisin de table.

Qu'est-ce qui vous semble ressortir ?

\subsection{Activité 2 : observation réfléchie (20 minutes)}

En petits groupes, vous identifiez une œuvre de science-fiction que vous connaissez.

\begin{itemize}
  \item Expliquez ce que vous avez ressenti au contact de cette œuvre
  \item Essayez d'expliquer pourquoi vous avez ressenti cela
  \item Comment la relation entre humain et technologie vous semble-t-elle avoir affecté votre perception des choses ?
\end{itemize}

\subsection{Activité 3 : conceptualisation (40 minutes)}

L'enseignant présente son cours.

Les étudiants :
\begin{itemize}
  \item Reprennent l'œuvre précédemment utilisée
  \item Analysent une interface de l'œuvre choisie en utilisant la grille 1
  \item Identifient les conséquences humaines et sociales de cette interface/technologie
  \item Utilisent la grille 2 pour jalonner leur réflexion
\end{itemize}

\subsection{Activité 4 : action (20 minutes)}

Toujours en groupe, et sur l'œuvre choisie :

Vous débattez et proposez une solution (technologique, réglementaire, éducative, etc.) permettant de résoudre les tensions éthiques abordées dans l'activité précédente.

\subsection{Conclusion (10 minutes)}

Synthèse collective et perspectives.

% ============================================
% RÉFÉRENCES
% ============================================

\section{Références}

\subsection{Ressources additionnelles}

\begin{itemize}
  \item Matsuda, K. (Réalisateur). (2016). \textit{Hyper Reality} [Vidéo]. YouTube. \url{https://www.youtube.com/watch?v=YJg02ivYzSs}
  \item Red Team. (s.d.). \textit{Le projet Red Team}. \url{https://redteamdefense.org/}
  \item \textit{SciFi Interfaces}. (s.d.). \url{https://scifiinterfaces.com/}
\end{itemize}

\subsection{Références pour l'étudiant}

\begin{itemize}
  \item Marcus, A. (2013). The history of the future: Sci-fi movies and HCI. \textit{Interactions}, \textit{20}(4), 64--67. \url{https://doi.org/10.1145/2486227.2486240}
  \item Raskin, J. (2000). \textit{The humane interface: New directions for designing interactive systems}. Addison-Wesley Professional.
  \item Shedroff, N., et Noessel, C. (2012). \textit{Make it so: Interaction design lessons from science-fiction}. Rosenfeld Media.
\end{itemize}

\subsection{Références pour l'enseignant}

\begin{itemize}
  \item Bruneault, F., Sabourin Laflamme, A., et Mondeaux, A. (2022). \textit{Former à l'éthique de l'IA en enseignement supérieur : Référentiel de compétence}.
  \item Bruneault, F., Sabourin Laflamme, A., Boivin, J., Grondin-Robillard, L., et Le Calvez, É. (2024). \textit{Former à l'éthique de l'IA en enseignement supérieur : Trousse pédagogique}.
  \item HEC Montréal. (s.d.). \textit{L'apprentissage expérientiel}. \url{https://enseigner.hec.ca/pedagogie/apprentissage-experientiel}
  \item Lacroix, A., Marchildon, A., et Bégin, L. (2017). \textit{Former à l'éthique en organisation : Une approche pragmatiste}. Presses de l'Université du Québec.
\end{itemize}

% ============================================
% GRILLES
% ============================================

\section{Documents pour les étudiants}

Ci-dessous :

\begin{itemize}
  \item \hyperlink{grille1}{Grille 1 : analyse d'interface}
  \item \hyperlink{grille2}{Grille 2 : analyse des tensions éthiques}
\end{itemize}

\newpage

\hypertarget{grille1}{}
\subsection{Grille 1 : analyse d'interface}

\textbf{Noms et prénoms des étudiants du groupe :}

\vspace{1em}

\textbf{Œuvre choisie :}

\vspace{1em}

\begin{longtable}{|>{\raggedright\arraybackslash}p{0.22\textwidth}|>{\raggedright\arraybackslash}p{0.23\textwidth}|>{\raggedright\arraybackslash}p{0.23\textwidth}|>{\raggedright\arraybackslash}p{0.23\textwidth}|}

\hline
\textbf{Technique / design} & \textbf{Industriel} & \textbf{Informationnel} & \textbf{Interactif} \\
\hline
\endhead

\hline
\endfoot

\hline
\endlastfoot

Contrôle mécanique / analogique & & & \\[2em]
\hline
Interfaces visuelles & & & \\[2em]
\hline
Interfaces sonores & & & \\[2em]
\hline
Mouvement et physiologie & & & \\[2em]
\hline
Réalité et virtualité & & & \\[2em]
\hline
Interface Cerveau-Machine & & & \\[2em]
\hline
Anthropomorphisme & & & \\[2em]
\hline
\end{longtable}

\vspace{2em}

\newpage

\textbf{Explications des catégories}

\begin{longtable}{|>{\raggedright\arraybackslash}p{0.22\textwidth}|>{\raggedright\arraybackslash}p{0.32\textwidth}|>{\raggedright\arraybackslash}p{0.18\textwidth}|>{\raggedright\arraybackslash}p{0.18\textwidth}|}

\hline
\textbf{Technique} & \textbf{Description} & \textbf{Design} & \textbf{Description} \\
\hline
\endhead

\hline
\endfoot

\hline
\endlastfoot

Contrôle mécanique / analogique & Interfaces mettant en œuvre des éléments physiques aussi bien pour les inputs que pour les outputs & Industriel & Conception des éléments physiques de l'interface \\
\hline

Interfaces visuelles & Interfaces disposant d'affichages visuels pouvant mettre en œuvre des solutions de type ligne de commande, interface graphique, métaphore du bureau & Informationnelle & Conception de la présentation de l'information de façon que la communication fasse du sens \\
\hline

Interfaces sonores & Interfaces utilisant les inputs et outputs sonores qui peuvent être de la voix, de la musique ou des « earcons » & Interactifs & Conception des séquences d'action et de rétroaction \\
\hline

Mouvement et physiologie & Interfaces capables de percevoir les mouvements du corps (déplacement des segments du corps) mais aussi des informations plus subtiles comme les mouvements oculaires ou les battements du cœur & & \\
\hline

Réalité et virtualité & Toutes les interfaces qui mélangent la sphère physique et cyber à différents degrés de la réalité virtuelle complète jusqu'aux hologrammes en passant par la réalité augmentée & & \\
\hline

Interface Cerveau-Machine & Interfaces qui utilisent les mesures d'activité cérébrales comme input voire comme commande & & \\
\hline

Anthropomorphisme & Interfaces qui prennent les apparences visuelles, vocales etc. de l'humain & & \\
\hline

\end{longtable}

\newpage

\hypertarget{grille2}{}
\subsection{Grille 2 : analyse des tensions éthiques}

\textbf{Noms et prénoms des étudiants du groupe :}

\vspace{1em}

\textbf{Œuvre choisie :}

\vspace{1em}

\begin{longtable}{|>{\raggedright\arraybackslash}p{0.38\textwidth}|>{\raggedright\arraybackslash}p{0.52\textwidth}|}
\hline
\textbf{Rubrique} & \textbf{Observation} \\
\hline
\endfirsthead
\hline
\textbf{Interface identifiée} & \\[2em]
\hline
\textbf{Mécanismes impliqués dans l'usage de cette technologie} & \\[3em]
\hline
\textbf{Conflit d'intérêts entre les promoteurs de l'interface et les utilisateurs finaux} & \\[3em]
\hline
\textbf{Dilemmes moraux et tension de valeurs que l'usage de cette technologie implique} & \\[3em]
\hline
\textbf{Anticipations des effets sur la société et les individus à différentes échelles de temps} & \\[3em]
\hline
\textbf{Cadre légal (réel, fictif, anticipé) pour réguler ce type de situation} & \\[3em]
\hline
\end{longtable}

\end{document}